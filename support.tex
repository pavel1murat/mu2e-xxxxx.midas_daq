%%%%%%%%%%%%%%%%%%%%%%%%%%%%%%%%%%%%%%%%%%%%%%%%%%%%%%%%%%%%%%%%%%%%%%%%%%%%%%
\section{Documentation and support considerations}

MIDAS has first been introduced about 30 years ago and has been used
by multiple experiments. It is used by the experiments at TRIUMF, PSI,
and CERN. Has been used by g-2. Currently is used by MEG, will be used by
the Mu3e experiment which will have
a close to Mu2e structure of the DAQ.
It evolved over time, its architecture is advanced and uses modern software
technologies (HTML5, Javascript, Ajax).
It is supported on Windows, Macs and Linux.
Expect active support to continue over the lifetime of Mu2e.

The MIDAS wiki \cite{2025_MIDAS_WIKI}, hosted by TRIUMF, provides an equivalent
of the user/developer manuals..

We expect the Mu2e members coming from g-2 to have experience with operating
a MIDAS-based DAQ.

\begin{itemize}
\item 
  MIDAS codebase is public and maintained at: https://bitbucket.org/tmidas/workspace/projects/PROJ
\item
  core developers : about 4-5 people
\item
  activity over the last 12 months: 100's of commits by the core developers/maintainers plus
  13 pull requests by 6 developers
\item 
  MIDAS forum - https://daq00.triumf.ca/elog-midas/Midas/
\end{itemize}




%%% Local Variables:
%%% mode: latex
%%% TeX-master: t
%%% End:
