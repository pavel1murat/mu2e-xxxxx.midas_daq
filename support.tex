%%%%%%%%%%%%%%%%%%%%%%%%%%%%%%%%%%%%%%%%%%%%%%%%%%%%%%%%%%%%%%%%%%%%%%%%%%%%%%
\section{Documentation and support considerations}

\begin{itemize}
\item
  MIDAS wiki \cite{2025_MIDAS_WIKI}, hosted by TRIUMF, serves the role of 
  of the user/developer manuals which is always up to date.
\item 
  The MIDAS forum provides an efficient way for communicating with the experts.
\item
  MIDAS is supported on Windows, Macs and Linux.
  Expect active support to continue over the lifetime of Mu2e.
\item 
  MIDAS codebase is public and maintained at: https://bitbucket.org/tmidas/workspace/projects/PROJ
\item
  significant developer base: about 4-5 core developers from several running experiments
\item
  activity over the last 12 months: 100's of commits by the core developers/maintainers plus
  13 pull requests by 6 developers
\end{itemize}


%%%%%%%%%%%%%%%%%%%%%%%%%%%%%%%%%%%%%%%%%%%%%%%%%%%%%%%%%%%%%%%%%%%%%%%%%%%%%%
\subsection{Software integration}
\begin{itemize}
\item 
  MIDAS-spack interface exists and MIDAS is described as a package in a spack-based system.
  Standard spack tools are used to build MIDAS libraries.
\item
  the system is in operation for more than 6 months.
\end{itemize}

%%% Local Variables:
%%% mode: latex
%%% TeX-master: t
%%% End:
