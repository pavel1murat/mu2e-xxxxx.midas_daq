\section{System architecture}

The proposed system consists of two big parts :
\begin{itemize}
\item
  ARTDAQ : handles the trigger farm and the data processing
\item
  MIDAS : handles the run configuration, run control, messaging,
  standardizes interfaces between various parts of the online system,
  and provides web-based interface for shifters,
\end{itemize}

%%%%%%%%%%%%%%%%%%%%%%%%%%%%%%%%%%%%%%%%%%%%%%%%%%%%%%%%%%%%%%%%%%%%%%%%%%%%%%
\subsection{MIDAS}

COre parts of MIDAS are :

\begin{itemize}
\item 
  mhttpd, a web server with restricted functionality
  mhttpd connects its clients to ODB. 
  The mhttpd doesn't execute any applications. Instead, it only updates the ODB.
\item
  Clients: frontends, communicating with the hardware and other
  parts of the system. Different types of frontends. Mu2e only needs slow monitoring.
\item
  Online DataBase (ODB) - a shared memory segment, which stores all configuration
  data of the system
  Despite its name, ODB is not a database. Instead, it stores only current configuration
  of the system, As the new run starts, a run number-stamped configuration is exported and stored.
  Logical organization of ODB maps onto a json structure and is well suited for describing
  complex configurations with multiple levels of hierarchy
\end{itemize}

MIDAS ODB has several interfaces:
\begin{itemize}
\item 
  web-based interface - via the mhttpd ODB page
\item
  command line interface (odbedit)
\item
  C++ interface allowing C++ clients to interact with ODB
\item
  python interface
\end{itemize}

In all cases, clients connect to MHTTPD and MHTTPD acts as
the only agent directly interacting with the ODB.


%%%%%%%%%%%%%%%%%%%%%%%%%%%%%%%%%%%%%%%%%%%%%%%%%%%%%%%%%%%%%%%%%%%%%%%%%%%%%%
\subsection{Frontends}

MIDAS supports frontends of different types, which include the data readout and monitoring
frontends.

MIDAS event building functionality is not used , done by ARTDAQ.

The architecture of the Mu2e DAQ requires only monitoring and slow control frontends.

All high bandwidth traffic goes through ARTDAQ.

%%%%%%%%%%%%%%%%%%%%%%%%%%%%%%%%%%%%%%%%%%%%%%%%%%%%%%%%%%%%%%%%%%%%%%%%%%%%%%
\subsection{Interprocess communication} 

Two mechanisms
\begin{itemize}
\item
  via ODB : some clients write into ODB, others read
\item
  separate collection of the monitoring information and its visualization
\item
  messaging: jsonrpc. Each client can have a jsonrpc server and receive messages
  from any other client.
\item
  clients can broadcast messages to the system (info and alarm)
\end{itemize}

%%%%%%%%%%%%%%%%%%%%%%%%%%%%%%%%%%%%%%%%%%%%%%%%%%%%%%%%%%%%%%%%%%%%%%%%%%%%%%
\subsection{Slow monitoring} 

Two mechanisms
\begin{itemize}
\item
  one monitoring frontend per node 
\item
  monitors DTCs, ROCs, and separate collection of the monitoring information and its visualization
\end{itemize}

%%%%%%%%%%%%%%%%%%%%%%%%%%%%%%%%%%%%%%%%%%%%%%%%%%%%%%%%%%%%%%%%%%%%%%%%%%%%%%
\subsection{Alarm system}

MIDAS has a built-in alarm system which could be used to inform shifters
about various categories of failures.

That is not a replacement for critical alarms coming from different inputs.
CRYO ? Accelerator ? - talk to Andy Hocker.

%%%%%%%%%%%%%%%%%%%%%%%%%%%%%%%%%%%%%%%%%%%%%%%%%%%%%%%%%%%%%%%%%%%%%%%%%%%%%% 
\subsection{Javascript web interface}

\begin{itemize}
\item 
  MIDAS web interface is a combination of HTML+Javascript.
\item 
  MIDAS provides Javascript API to ODB, which facilitates development of
  functional experiment-specific web pages.
\end{itemize}


%%% Local Variables:
%%% mode: latex
%%% TeX-master: t
%%% End:


