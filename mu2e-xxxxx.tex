% -*- mode:flyspell; mode:latex -*-
\documentclass[12pt]{article}
\addtolength{\oddsidemargin} {-0.885in}
\addtolength{\textwidth}{1.75in}
\addtolength{\evensidemargin}{-0.8in}


\usepackage[latin1]{inputenc}
\usepackage[T1]{fontenc}
\usepackage[english]{babel}
\usepackage{graphicx}
\usepackage{float}
%% \usepackage{siunitx}

%% \usepackage{gensymb}


\usepackage{tikz}
\usepackage{[caption}
\usetikzlibrary{arrows}
\usetikzlibrary{decorations.markings}
\usetikzlibrary{decorations.pathmorphing}
% \usepackage[absolute,overlay]{textpos}
% \usepackage{onimage}

\usepackage{tabularx}
\usepackage{times}
\usepackage{graphics}

% \usepackage{subfigure}
% \usepackage{scalefnt}
%
% \renewcommand\thesubfigure{\arabic{subfigure}}

\usepackage{amsmath}
\usepackage{hyperref}
\usepackage{hhline}
\usepackage{subfig}
\usepackage{color}
\usepackage[all]{hypcap}

\usepackage[normalem]{ulem}  % for striking out
\usepackage{soul}
% \usepackage{fancyhdr}
% \pagestyle{fancy}
% \fancyhead[C]{}
% \fancyhead[L] {\it{Mu2e-doc-29670-v1.0} }
%%%%%%%%%%%%%%%%%%%%%%%%%%%%%%%%%%%%%%%%%%%%%%%%%%%%%%%%%%%%%%%%%%%%%%%%%%%%%%
% use natbib - biblatex not available on Mu2e interactive nodes
%%%%%%%%%%%%%%%%%%%%%%%%%%%%%%%%%%%%%%%%%%%%%%%%%%%%%%%%%%%%%%%%%%%%%%%%%%%%%%
\usepackage[square,sort,comma,numbers]{natbib}

% location of the .bib files: env var BIBINPUTS (~/library/bibliography)

% \usepackage[backend=biber, style=numeric-comp, sorting=ynt] {biblatex}
% \addbibresource{clfv.bib}

% \addbibresource{stntuple.bib}
% \addbibresource{mu2e_web.bib}
% \addbibresource{radiative_pion_capture.bib}

\graphicspath{{figures/}}
%%%%%%%%%%%%%%%%%%%%%%%%%%%%%%%%%%%%%%%%%%%%%%%%%%%%%%%%%%%%%%%%%%%%%%%%%%%%%%
% for portability, make sure all commands are included locally,
%%%%%%%%%%%%%%%%%%%%%%%%%%%%%%%%%%%%%%%%%%%%%%%%%%%%%%%%%%%%%%%%%%%%%%%%%%%%%%
\definecolor{ForestGreen}{RGB}{20,109,20}
%%%%%%%%%%%%%%%%%%%%%%%%%%%%%%%%%%%%%%%%%%%%%%%%%%%%%%%%%%%%%%%%%%%%%%%%%%%%%%%
\newcommand {\abseta}    {\mbox{$\mid \eta \mid $}}
\newcommand {\Au}[1]     {\mbox{$\rm ^{#1}Au$}}          % isotopes of gold

\newcommand {\baftwo}    {\mbox{$BaF_2$}}
\newcommand {\blue}      {\color{blue}}

\newcommand {\deltaE}    {\mbox{$\delta{\rm-electron}$}}

\newcommand {\eemm}      {\mbox{$ee\mu\mu$}}
\newcommand {\emfr}      {\mbox{$\epsilon_{EM}$}}
\newcommand {\eminus}    {\mbox{$e^-$}}
\newcommand {\eplus}     {\mbox{$e^+$}}
\newcommand {\et}        {\mbox{$E_t$}}
\newcommand {\etcorr}    {\mbox{$E_T^{corr}$}}
\newcommand {\etaa}      {\mbox{$\eta$}}

\newcommand {\green}     {\color{ForestGreen}}
\newcommand {\gt}        {\mbox{$>$}}
\newcommand {\gea}       {\mbox{$>=$}}
\newcommand {\gevcsq}    {\mbox{$GeV\!/c^2$}}

\newcommand {\invfb}     {\mbox{$fb^{-1}$}}
\newcommand {\invfbrm}   {\mbox{$\rm fb^{-1}$}}
\newcommand {\invpb}     {\mbox{$pb^{-1}$}}
\newcommand {\invpbrm}   {\mbox{$\rm pb^{-1}$}}
\newcommand {\Ir}[1]     {\mbox{$\rm ^{#1}Ir$}}                 % isotopes of iridium

\newcommand {\jpsi}      {\mbox{$J/\psi$}}

\newcommand {\keVc}      {\mbox{$\rm keV\!/\!c$}}
\newcommand {\kmax}      {\mbox{$k_{\rm max}$}}

\newcommand {\lea}       {\mbox{$<=$}}
\newcommand {\lt}        {\mbox{$<$}}

\newcommand {\mcluster}  {\mbox{$M_{cluster}$}}
\newcommand {\met}       {\mbox{${\not\! E}_{T}$}}
\newcommand {\MeVc}      {\mbox{$\rm MeV\!/c$}}
\newcommand {\MeVcsq}    {\mbox{$\rm MeV\!/c^2$}}
\newcommand {\mmmm}      {\mbox{$\mu\mu\mu\mu$}}
\newcommand {\mt}        {\mbox{$E_{T}$ }}
\newcommand {\mtrkpi}    {\mbox{${M}_{trk+{\pi^0}'s}$ }}

% \newcommand {\mumemconv} {\mbox{$\mu^- A \rightarrow e^- A$}}
% \newcommand {\mumepconv} {\mbox{$\mu^- A \rightarrow e^+ A$}}

\newcommand \mumepconvRelay[1][A]  {\mbox{$\mu^- \textrm{\ArgI} \rightarrow e^+ \textrm{#1}$}}
\newcommand {\MuToEm}     {\mbox{$\mu^- \ra e^-$}}
\newcommand {\MuToEp}     {\mbox{$\mu^- \ra e^+$}}
\newcommand {\MuPToEp}    {\mbox{$\mu^+ \ra e^+$}}

\newcommand {\mumemconv}[1][A] {\mbox{$\mu^- \textrm{#1} \rightarrow e^- \textrm{#1}$}}
% Define a relay to have 2 default arguments instead of limit of 1
\newcommand {\mumepconv}[1][A] {%
  \def\ArgI{{#1}}%store the first argument
  \mumepconvRelay
}

\newcommand {\mzz}       {\mbox{$M_{ZZ}$}}

\newcommand {\pb}        {\mbox{\rm\,pb}}
\newcommand {\Pb}[1]     {\mbox{$\rm ^{#1}Pb$}}           % isotopes of lead: \Pb[208]

\newcommand {\pbar}      {\mbox{$\bar{p}$}}
\newcommand {\pizero}    {\mbox{$\pi^0$}}
\newcommand {\piplusenu} {\mbox{$\pi^+ \to e^+ \nu$}}
\newcommand {\pfake}     {\mbox{$P_{fake}$}}
\newcommand {\ppbar}     {\mbox{$p\bar{p}$}}
\newcommand {\ppbarw}    {\mbox{$p\bar{p} \rightarrow W$}}
\newcommand {\pt}        {\mbox{$P_t$ }}
\newcommand {\purple}    {\color{purple}}

\newcommand {\ra}        {\mbox{$\rightarrow$}}
\newcommand {\red}       {\color{red}}
\newcommand {\Rmue}      {\mbox{$R_{\mu e}$}}

\newcommand {\stat}          {\mbox{\rm (stat.)}}
\newcommand {\statsys}       {\mbox{\rm (stat.+syst.)}}
\newcommand {\syst}          {\mbox{\rm (syst.)}}

\newcommand {\tauenunu}  {\mbox{$\tau \ra e \nu \bar{\nu}$}}
\newcommand {\taumununu} {\mbox{$\tau \ra \mu \nu \bar{\nu}$}}
\newcommand {\tandip}    {\mbox{$\tan \lambda$}}
\newcommand {\ttau}      {\mbox{$\tau$}}
\newcommand {\Tau}       {\mbox{$\tau$}}

\newcommand {\violet}    {\color{violet}}
\newcommand {\vprop}     {\mbox{$v_{prop}$}}

\newcommand {\upar}      {\mbox{$u_{||}$}}
\newcommand {\uperp}     {\mbox{$u_{perp}$}}

\newcommand {\wenu}      {\mbox{$W\rightarrow e\nu$}}
\newcommand {\wlnu}      {\mbox{$W\rightarrow l\nu$}}
\newcommand {\wmunu}     {\mbox{$W\rightarrow \mu \nu$}}
\newcommand {\wpigamma}  {\mbox{$W^{\pm} \rightarrow \pi^{\pm} \gamma$}}
\newcommand {\wtaunu}    {\mbox{$W\rightarrow\tau\nu$}}

\newcommand {\zee}       {\mbox{$Z \rightarrow e^{+}e^{-}$}}
\newcommand {\zll }      {\mbox{$Z       \rightarrow l^{+}l^{-}$}}
\newcommand {\zmumu}     {\mbox{$Z \rightarrow \mu^{+}\mu^{-}$}}
\newcommand {\znunu}     {\mbox{$Z \rightarrow \nu\nu$}}
\newcommand {\zpsigamma} {\mbox{$    Z^{0} \rightarrow J/\psi \gamma$}}
\newcommand {\zpigamma}  {\mbox{$\rm Z^{0} \rightarrow \pi^{0} \gamma$}}
\newcommand {\ztautau}   {\mbox{$Z \rightarrow \tau\tau$}}
\newcommand {\zupsgamma}     {\mbox{$    Z^0 \rightarrow \Upsilon \gamma$}}
\newcommand {\zzx }          {\mbox{$X       \to ZZ$}}
\newcommand {\zzllll}        {\mbox{$ZZ \to \ell^+ \ell^- \ell^+ \ell^-$}}
\newcommand {\zzllnn}        {\mbox{$ZZ \to \ell^+ \ell^- \nu \nu$}}
\newcommand {\zzlljj}        {\mbox{$ZZ \to \ell^+ \ell^- j j$}}
%%%%%%%%%%%%%%%%%%%%%%%%%%%%%%%%%%%%%%%%%%%%%%%%%%%%%%%%%%%%%%%%%%%%%%%%%%%%%%
% editing commands
%%%%%%%%%%%%%%%%%%%%%%%%%%%%%%%%%%%%%%%%%%%%%%%%%%%%%%%%%%%%%%%%%%%%%%%%%%%%%%
\newcommand {\add}[1]    {{\red         {#1}}}
\newcommand {\strike}[1] {{\blue   \sout{#1}}}
\newcommand {\del}[1]    {{\blue   \sout{#1}}}
\newcommand {\dlt}[1]    {{\violet \sout{#1}}}   %alternate delete color
\newcommand {\hlite}[1]  {{\red      \ul{#1}}}   % needs xcolor, soul
%%%%%%%%%%%%%%%%%%%%%%%%%%%%%%%%%%%%%%%%%%%%%%%%%%%%%%%%%%%%%%%%%%%%%%%%%%%%%%%

% %%%%%%%%%%%%%%%%%%%%%%%%%%%%%%%%%%%%%%%%%%%%%%%%%%%%%%%%%%%%%%%%%%%%%%%%%%%%%%
% for editors
% %%%%%%%%%%%%%%%%%%%%%%%%%%%%%%%%%%%%%%%%%%%%%%%%%%%%%%%%%%%%%%%%%%%%%%%%%%%%%%
\newcommand {\pasha}[1]     {{\green  #1}}
\newcommand {\simon}[1]     {{\blue   #1}}
%%%%%%%%%%%%%%%%%%%%%%%%%%%%%%%%%%%%%%%%%%%%%%%%%%%%%%%%%%%%%%%%%%%%%%%%%%%%%%
\begin{document}

\begin{titlepage}
  \begin{flushright}
    \bf {MU2E/PHYSICS/50071} \\
    version 1.02
    \today
 \end{flushright}

  \vspace{1cm}

  \begin{center}
    {\Large \bf MIDAS+ARTDAQ-based data acquisition system for Mu2e
      \vspace{0.3in}
    }

    \vspace{1cm}
    S.Corrodi(ANL), P.Murat(FNAL), ...

    % \footnote{\texttt{Fermilab; e-mail: murat@fnal.gov}}
    \vspace{0.3cm}

    \vspace{0.8cm}
  \end{center}

  \begin{abstract}
    \vspace{0.2in}

    This document describes a solution for the Mu2e data acquisition
    system based on MIDAS+ARTDAQ {\red should there be a name? }.

    It has been prototyped and tested by several Mu2e
    detector groups. 

    We propose to exercise this solution and demonstrate
    its functionality in the upcoming GR4.
  \end{abstract}

\end{titlepage}
% \frontmatter
% \chapter*{Abstract}
%
% \addcontentsline{toc}{chapter}{Abstract}
%
% \mainmatter
%
{\tableofcontents}

%%%%%%%%%%%%%%%%%%%%%%%%%%%%%%%%%%%%%%%%%%%%%%%%%%%%%%%%%%%%%%%%%%%%%%%%%%%%%%%
%\chapter{Calibration}
%%%%%%%%%%%%%%%%%%%%%%%%%%%%%%%%%%%%%%%%%%%%%%%%%%%%%%%%%%%%%%%%%%%%%%%%%%%%%%%
% \input{input_data}

%%%%%%%%%%%%%%%%%%%%%%%%%%%%%%%%%%%%%%%%%%%%%%%%%%%%%%%%%%%%%%%%%%%%%%%%%%%%%%%
\newpage
\section {Revision History and TODO items}

\begin{itemize}
\item
  v1.01: version for internal review
\end{itemize}

% {\red
% TODO items:
% 
% \begin{itemize}
% \item
% \end{itemize}
% }
%%%%%%%%%%%%%%%%%%%%%%%%%%%%%%%%%%%%%%%%%%%%%%%%%%%%%%%%%%%%%%%%%%%%%%%%%%%%%%
\newpage
\section {Executive Summary}

Development of the Mu2e DAQ has been lagging behind for quite significant time.
Despite some recent progress, after years of trying no Mu2e collaborator
can operate the DAQ system in their own. With a few months left before the start of the
cosmic run, the system has missing pieces which are yet to be designed.
%
The core of the DAQ system, OTSDAQ, has not been used before by any running experiment.
In is complicated and undocumented. The interfaces between different parts are not
specified, and its complexity and lack of modularity create a very high entry
threshold for the Mu2e collaborators.
%
Most of the DAQ development is happening outside the collaboration which has very
limited input into the development process.
%
During the data taking, the collaboration will have to operate the DAQ system
and it will have to be functional 24x7. In our view, the cost of operating 
and supporting the present DAQ system, expressed in terms of the lost beam time,
will be too high.

% 
We have therefore developed and prototyped a DAQ system based on the combination of two
packages, ARTDAQ \cite{2017_ARTDAQ_Biery} and MIDAS \cite{2025_MIDAS_WIKI}.
%
ARTDAQ is a distributed trigger farm software developed at Fermilab and previously
used by several Fermilab neutrino experiments. 
%
MIDAS is a full DAQ suite which has been used for about 30 years by many HEP experiments
at PSI, TRIUMF, CERN, KEK. It should be noted that MIDAS has also been used
by the g-2 experiment at Fermilab.
%
Out of the currently running experiments, MIDAS is currently used by MEG-II which,
as Mu2e,  searches for CLFV. Another CLFV experiment, Mu3e, which DAQ hardware architecture
is rather close to that of Mu2e, will also be using a MIDAS-based DAQ system.
In a way, MIDAS became a de-facto DAQ standard for small and medium scale HEP
experiments.

Any DAQ system at its early stages has many issues which can only be uncovered and fixed
by operating it. MIDAS has been debugged and tuned by tens of experiments
and by decades of the running time. Although the initial design of MIDAS goes 30 years back,
its architecture is advanced, and the software infrastructure is constantly updated
to use modern technologies.

Based on our technical assessment, a system, which combines the power of
the distributed ARTDAQ data processing with the strength of the integrated
MIDAS-based run control and monitoring, provides the best technical solution
for the Mu2e DAQ.
%
Adding to that, many Mu2e collaborators already have operational experience with
the MIDAS-based DAQ. New collaboration members, coming to Mu2e from other muon 
experiments will likely know MIDAS.

The following sections describe the architecture of the proposed system,
its interfaces, and the status of the present implementation.

\section{System architecture}

The proposed system consists of two major software components :
\begin{itemize}
\item
  ARTDAQ : handles the trigger farm and the data processing
\item
  MIDAS : handles the run configuration, run control, messaging,
  standardizes interfaces between various parts of the online system,
  and provides web-based interfaces for shifters and experts,
\end{itemize}

ARTDAQ supports computing farms with distributed architecture, 
and all high-bandwidth data traffic goes through ARTDAQ.
%
MIDAS architecture assumes a central server and multiple clients,
best suited for configuration/control applications.

The architecture of the proposed DAQ system is schematically presented
in Figure ~\ref{figure:system_architecture} and explained in the remaining
part of this section.

\begin{figure}[H]
  \begin{tikzpicture}
    \node[anchor=south west,inner sep=0] at (0,0.) {
      % \node[shift={(0 cm,0.cm)},inner sep=0,rotate={90}] at (0,0) {}
      \makebox[\textwidth][c] {
        \includegraphics[width=1.1\textwidth]{pdf/system_architecture}
      }
    };
    % \node [text width=8cm, scale=1.0] at (14.5,0.5) {$\mu_B$, expected background mean};
    % \node [text width=8cm, scale=1.0, rotate={90}] at (1.5,7.5) { $S_{D}$, ``discovery'' signal strength  };
  \end{tikzpicture}
  \caption{
    \label{figure:system_architecture}
    Architecture of the MIDAS-based DAQ system. Communication protocols:
    red solid lines - JSON-RPC, blue solid lines - XML-RPC,
    red dashed lines - http(s).
  }
\end{figure}

%%%%%%%%%%%%%%%%%%%%%%%%%%%%%%%%%%%%%%%%%%%%%%%%%%%%%%%%%%%%%%%%%%%%%%%%%%%%%%
\subsection{MIDAS}

The core parts of MIDAS are :

\begin{itemize}
\item
  {\bf Online DataBase (ODB)} : a shared memory segment, which stores all configuration data of the system.
  Despite its name, ODB is not a database. Instead, it stores only current configuration
  of the system, As the new run starts, a run number-stamped configuration is exported and stored.
  Logical organization of ODB maps onto a json structure and is well suited for describing
  complex run and detector configurations with multiple levels of hierarchy
\item 
  {\bf mhttpd}: a web server with restricted functionality
  mhttpd connects its clients to ODB. 
  The mhttpd doesn't execute any applications. Instead, it only updates the ODB.
\item
  {\bf mlogger}: an executable responsible for logging the data. In case of Mu2e, that is the slow control data.
  mlogger supports multiple backends, including local files, MySQL and PGSDL databases
\item
  {\bf mserver}: an executable handling communication with remote clients
\item
  {\bf sequencer}: an interactive tool handling the detector hardware configuration.
  The Sequencer has interfaces to ODB, operating system, and a simple scripting language
  which provides the level of flexibility needed for configuring a detector of any complexity
  at begin run.
\item
  clients: frontends, communicating with the hardware and other
  parts of the system. Different types of frontends. Mu2e only needs slow monitoring.
\item
  MIDAS state machine has four transitions: Start, Stop, Pause, and Resume.
  Combined with the Sequencer-based hardware configuration, that provides everything
  necessary for the DAQ state machine.
\end{itemize}

%%%%%%%%%%%%%%%%%%%%%%%%%%%%%%%%%%%%%%%%%%%%%%%%%%%%%%%%%%%%%%%%%%%%%%%%%%%%%%
\subsection{ODB and its interfaces}

MIDAS ODB has several interfaces:
\begin{itemize}
\item 
  web-based interface - via the mhttpd "ODB" page
\item
  odbedit: a command line interface
\item
  C/C++ interface
\item
  python interface
\end{itemize}

In all cases, clients connect to MHTTPD and MHTTPD acts as
the only agent directly interacting with the ODB.

%%%%%%%%%%%%%%%%%%%%%%%%%%%%%%%%%%%%%%%%%%%%%%%%%%%%%%%%%%%%%%%%%%%%%%%%%%%%%%
\subsection{Frontends}

\begin{itemize}
\item 
  MIDAS supports frontends of different types, which include the data readout and monitoring
  frontends.
\item 
  MIDAS event building functionality is not used, event building is performed by ARTDAQ.
\item 
  The architecture of the Mu2e DAQ requires only monitoring and slow control frontends.
\item
  frontends can be implemented in C, C++, and Python. The frontend templates are provided
  with the distribution.
\end{itemize}


%%%%%%%%%%%%%%%%%%%%%%%%%%%%%%%%%%%%%%%%%%%%%%%%%%%%%%%%%%%%%%%%%%%%%%%%%%%%%%
\subsection{Interface to ARTDAQ}
ARTDAQ handles all high-bandwidth data traffic, which includes reading the
detector data, one-level software trigger, data logging, and event distribution
to online consumers (DQM).
%
ARTDAQ processes are controlled by the  \href{https://github.com/pavel1murat/frontends/blob/main/tfm_frontend/tfm_launch_fe.py}
{\blue Trigger Farm Manager (TFM) frontend}, which interfaces MIDAS to the
\href{https://github.com/pavel1murat/tfm/blob/main/rc/control/farm_manager.py}{\blue TFM}.

\begin{itemize}
\item 
  TFM reads its initial settings from the "/Mu2e/ActiveRunConfiguration/DAQ/Tfm"
  ODB configuration path.
\item
  FCL files of ARTDAQ processes are stored in \$MU2E\_DAQ\_DIR/config/\$run\_configuration
  subdirectory assumed to be the same (network mounted) on all DAQ nodes.
  The FCL files could be regenerated manually based on the run configuration and the trigger table.
\item
  Archival mechanism: the FCL files used for a given run are stored in \\
  \$DAQ\_OUTPUT\_TOP/run\_records/\$run\_number area of the node where the TFM frontend is running.
\item
  the log files of ARTDAQ processes running on a given node, one logfile per process per run,
  are stored in \$DAQ\_OUTPUT\_TOP/logs directory of that node
\end{itemize}


%%%%%%%%%%%%%%%%%%%%%%%%%%%%%%%%%%%%%%%%%%%%%%%%%%%%%%%%%%%%%%%%%%%%%%%%%%%
\subsubsection{Communication with ARTRAQ}

\begin{itemize}
\item
  native artdaq communication is internal to the ARTDAQ
\item
  ARTDAQ processes report their metrics in XML-RPC format,
  and it is assumed that parsing of the metrics, visualizing
  it and generating alarms is the job of the 3-rd party application,
  i.e. {\bf grafana}.
\item
  TFM queries the ARTDAQ process metrics over XML-RPC and records
  them in ODB. That obsoletes the need in the 3rd party applications.
\item
  Mu2e -specific applications, i.e. the boardreaders and the filter modules 
  communicate to the MIDAS server directly via messaging and report 
  their status. Therefore simple alarms like single ROC timeouts etc
  could be generated as they are detected. 
  This avoids a need in an additional software to recognize an alarming
  situation and trigger the corresponding notification mechanism.
\end{itemize}


%%%%%%%%%%%%%%%%%%%%%%%%%%%%%%%%%%%%%%%%%%%%%%%%%%%%%%%%%%%%%%%%%%%%%%%%%%%%%%
\subsection{Interface to the online run conditions database} 

\begin{itemize}
\item
  the interface is provided by the
  \href{https://github.com/pavel1murat/frontends/blob/main/utils/runinfodb.py}
  {\blue python module talking the PG database}, MIDAS Sequencer, and the
  \href{https://github.com/pavel1murat/frontends/blob/main/conf/mu2e_config_fe.py}
  {\blue global configuration frontend}. The sequence is as follows:
\item
  the Sequencer requests  a new run number from the run configuration DB
  and stores it in ODB as  {\blue"/Runinfo/Run number"}.
\item
  during the begin run transition, the TFM reads the run number from ODB
  and passes it to ARTDAQ processes via the environment - env var {\blue RUN\_NUMBER}
\item
  the global configuration frontend registers the start and the stop of each
  run transition in the run configuration database (to be checked).
\end{itemize}


%%%%%%%%%%%%%%%%%%%%%%%%%%%%%%%%%%%%%%%%%%%%%%%%%%%%%%%%%%%%%%%%%%%%%%%%%%%%%% 
\subsection{Slow monitoring and interface to EPICS}

The slow monitoring framework is provided by the MIDAS history system.
THis is one of the best technical solutions to the slow monitoring problem possible.
The interface via ODB is simple and crisp:
\begin{itemize}
\item
  all parameters to be monitored are stored by the monitoring frontends in ODB.
\item
  the visualization utilities simply display the content of the predefined ODB locations.
\end{itemize}

Therefore, the visualization and control is completely decoupled from the collection of the
monitoring parameters.

The historical data could be stored in  MIDAS internal format or in a database.
The list of supported databases includes ODBC, SQLITE, MYSQL and PGSQL.

This functionality is similar to that of grafana, however integrated with the
run control system

\begin{itemize}
\item
  one monitoring frontend per node 
  \begin{itemize}
  \item
    controls and configures DTC's and ROCs on this node
  \item
    controls the CFO if the CFO is running on this node
  \item
    monitors DTCs and ROCs, collecting history and non-history information
  \item
    monitors artdaq processes d separate collection of the monitoring information and its visualization
  \end{itemize}
\end{itemize}

Figure ~\ref{figure:slow_controls_node_page} gives an example of one of the slow controls pages

\begin{figure}[H]
  \begin{tikzpicture}
    \node[anchor=south west,inner sep=0] at (0,0.) {
      % \node[shift={(0 cm,0.cm)},inner sep=0,rotate={90}] at (0,0) {}
      \makebox[\textwidth][c] {
        \includegraphics[width=0.95\textwidth]{png/slow_controls_node_page}
      }
    };
    % \node [text width=8cm, scale=1.0] at (14.5,0.5) {$\mu_B$, expected background mean};
    % \node [text width=8cm, scale=1.0, rotate={90}] at (1.5,7.5) { $S_{D}$, ``discovery'' signal strength  };
  \end{tikzpicture}
  \caption{
    \label{figure:slow_controls_node_page}
    A prototype of the DAQ node slow control page. Monitored are the DTC and ROC temperatures and voltages
    and rates of the ARTDAQ processes
  }
\end{figure}


In addition to its own fully implemented history system, MIDAS also has an interface to EPICS.
The interface is implemented as a 
\href{https://bitbucket.org/tmidas/midas/src/develop/examples/epics/}
         {\blue slow control frontend} with the 
\href{https://bitbucket.org/tmidas/midas/src/develop/drivers/device/epics_ca.cxx}
{\blue special EPICS driver}.
%
One therefore can in a transparent way combine EPICS data collection capabilities 
with the MIDAS web display system. MEG-II uses this approach for the beam parameter monitoring.

%%% Local Variables:
%%% mode: latex
%%% TeX-master: t
%%% End:


%%%%%%%%%%%%%%%%%%%%%%%%%%%%%%%%%%%%%%%%%%%%%%%%%%%%%%%%%%%%%%%%%%%%%%%%%%%%%%
\subsection{Alarm system}

MIDAS has a built-in alarm system which could be used to inform shifters
about various categories of failures.

That is not a replacement for critical alarms coming from different inputs.
CRYO ? Accelerator ? -{\red talk to Andy Hocker.}


%%%%%%%%%%%%%%%%%%%%%%%%%%%%%%%%%%%%%%%%%%%%%%%%%%%%%%%%%%%%%%%%%%%%%%%%%%%%%% 
\subsection{Javascript web interface}

\begin{itemize}
\item 
  MIDAS web interface is a combination of HTML5+Javascript, based on the
  asynchronous approach to the web pages update (Ajax).
\item 
  MIDAS provides Javascript API to ODB, which facilitates development of
  functional experiment-specific {\bf custom} web pages.
  The interface is documented at \\
  \href{https://daq00.triumf.ca/MidasWiki/index.php/Custom\_Page}
  {\blue https://daq00.triumf.ca/MidasWiki/index.php/Custom\_Page}
\end{itemize}


%%%%%%%%%%%%%%%%%%%%%%%%%%%%%%%%%%%%%%%%%%%%%%%%%%%%%%%%%%%%%%%%%%%%%%%%%%%%%%
\subsection{Interfaces to ECL, ELOG, Slack}

MIDAS has interfaces to internal and external ELOGs and Slack.
\begin{itemize}
\item
  interface to the external ELOG has been implemented and tested
\item
  interface to Slack has been tested and used by at least two experiments - MEG and g-2
\item
  interface to ECL - to be implemented
\end{itemize}


%%%%%%%%%%%%%%%%%%%%%%%%%%%%%%%%%%%%%%%%%%%%%%%%%%%%%%%%%%%%%%%%%%%%%%%%%%%%%%
\subsection{DQM}

DQM processes use ROOT-based histogramming. As any ROOT executable is a web server,
the DAQM jobs publish their histograms on the web, and their clients (users)
simply connect to them over the http protocol.

%%%%%%%%%%%%%%%%%%%%%%%%%%%%%%%%%%%%%%%%%%%%%%%%%%%%%%%%%%%%%%%%%%%%%%%%%%%%%% 
\subsection{Interprocess communication} 

The system supports three mechanisms
\begin{itemize}
\item
  via ODB : some clients write into ODB, others read
\item
  separate collection of the monitoring information and its visualization
\item
  MIDAS messaging: jsonrpc.
  \begin{itemize}
  \item
    Each client can have a jsonrpc server and receive messages
    from any other client.
  \item
    clients can broadcast messages to the system (info and alarm)
    and communicate to each other
  \item
    each frontend, C++ or Python, has a built-in jsonrpc communication
    functionality built-in.
  \end{itemize}
\item
  communication between ARTDAQ processes is based on an older technology,
  XML-RPC. The TFM frontend supports the XML-RPC-based ARTDAQ communication
  protocol.
\end{itemize}


%%%%%%%%%%%%%%%%%%%%%%%%%%%%%%%%%%%%%%%%%%%%%%%%%%%%%%%%%%%%%%%%%%%%%%%%%%%%%% 
\subsection{Data model ??? } 




%%%%%%%%%%%%%%%%%%%%%%%%%%%%%%%%%%%%%%%%%%%%%%%%%%%%%%%%%%%%%%%%%%%%%%%%%%%%%%
\subsection{DAQ}
Top view of the DAQ configuration hierarchy is shown in Figure~\ref{figure:daq_config}.

{ \scriptsize
\begin{verbatim}
[local:test_025:S]DAQ>ls -l  
Key name                        Type    #Val  Size  Last Opn Mode Value
---------------------------------------------------------------------------
CFO                             DIR
Subsystems                      DIR
Nodes                           DIR
Tfm                             DIR
# --- artdaq parameters ----    STRING  1     65    12m  0  RWD   ------ first_port = BasePortNumber+PartitionID*PortsPerPartition
PartitionID                     INT32   1     4     12m  0  RWD   11
PortsPerPartition               INT32   1     4     12m  0  RWD   1000
BasePortNumber                  INT32   1     4     12m  0  RWD   10000
#1 ----------------------       STRING  1     64    12m  0  RWD   LocalSubnet and PublicSubnet are used to resolve the host names
PrivateSubnet                   STRING  1     32    12m  0  RWD   131.225.38
PublicSubnet                    STRING  1     32    12m  0  RWD    131.225.38
MIDAS_SERVER_HOST               STRING  1     32    12m  0  RWD   mu2edaq09
# ---- assumption:              STRING  1     67    12m  0  RWD   OnSpill, RocReadoutMode, and EventMode are the same for all DTCs
OnSpill                         INT32   1     4     12m  0  RWD   1
RocReadoutMode                  INT32   1     4     12m  0  RWD   1
EventMode                       INT32   1     4     12m  0  RWD   1
SkipDtcInit                     INT32   1     4     12m  0  RWD   0
# --------------------- 0       STRING  1     42    12m  0  RWD   -------------------- configurable element
MonitoringFrontend              DIR
Enabled                         INT32   1     4     12m  0  RWD   1
Status                          INT32   1     4     12m  0  RWD   0
\end{verbatim}

}

\begin{itemize}
\item
  {\bf CFO} : subdirectory, contains the CFO (Central FanOut module) parameters.
  In the running configuraton there could be only one CFO.
  The CFO parameters are described in subsection \ref{section:cfo}
\item
  {\bf Subsystems} : subdirectory, describes ARTDAQ subsystems
\item
  {\bf Nodes} : subdirectory, describes configurations of the individual trigger farm nodes
\item
  {\bf Tfm} : subdirectory, describes the configuration of the Trigger Farm Manager(TFM)
\end{itemize}


\begin{figure}[H]
  \begin{tikzpicture}
    \node[anchor=south west,inner sep=0] at (0,0.) {
      % \node[shift={(0 cm,0.cm)},inner sep=0,rotate={90}] at (0,0) {}
      \makebox[\textwidth][c] {
        \includegraphics[width=0.95\textwidth]{png/daq_configuration}
      }
    };
    % \node [text width=8cm, scale=1.0] at (14.5,0.5) {$\mu_B$, expected background mean};
    % \node [text width=8cm, scale=1.0, rotate={90}] at (1.5,7.5) { $S_{D}$, ``discovery'' signal strength  };
  \end{tikzpicture}
  \caption{
    \label{figure:daq_config}
    Top level of the DAQ configuration
  }
\end{figure}

It includes configuration of the CFO, the Trigger Farm Manager (TRM), configuration
of the trigger farm nodes and a number of global DAQ parameters.


%%%%%%%%%%%%%%%%%%%%%%%%%%%%%%%%%%%%%%%%%%%%%%%%%%%%%%%%%%%%%%%%%%%%%%%%%%%%%%
\newpage
\subsubsection{CFO configuration}
\label{section:cfo}

Emulated CFO is controlled by the ``emulated CFO frontend''.
That is a frontend with the CFO described a s MIDAS ``periodic equipment''.
When the frontend's periodic readout function is called, a new train of EWMs
is sent to the DTCs.

An example of the emulated CFO configuration is shown in Figure~\ref{figure:cfo_config}.
It includes a link to the configuration of the corresponding FPGA (DTC) 
defined within the same configuration, and a list of the CFO-specific parameters.

\begin{figure}[H]
{ \scriptsize
\begin{verbatim}
[local:test_025:S]CFO>ls -l  
Key name                        Type    #Val  Size  Last Opn Mode Value
---------------------------------------------------------------------------
EmulatedMode                    INT32   1     4     19m  0  RWD   1
Host                            STRING  1     32    19m  0  RWD   mu2edaq09.fnal.gov
NEventsPerTrain                 INT32   1     4     19m  0  RWD   200
EventWindowSize                 INT32   1     4     19m  0  RWD   100
SleepTimeMs                     INT32   1     4     19m  0  RWD   250
DTC -> /Mu2e/RunConfigurations/train_station/DAQ/Nodes/mu2edaq09/DTC0
                                KEY     1     12    >99d 0  RWD   <subdirectory>
Enabled                         INT32   1     4     19m  0  RWD   1
Status                          INT32   1     4     19m  0  RWD   0
\end{verbatim}
}
\caption{
  \label{figure:cfo_config}
  CFO configuration
}
\end{figure}

\begin{itemize}
\item
  {\bf EmulatedMode} : 1: CFO is emulated by a DTC 
  0: external CFO (a separate FPGA)
\item
  {\bf NEventsPerTrain} : number of non-null EWMs in the train
\item
  {\bf EventWindowSize} : event window in units of 25 ns
\item
  {\bf SleepTimeMs}     : the CFO frontend operates as follows: it generates a train of (HB+EWM)'s
  which ends with a null heartbeat, then sleeps for {\bf SleepTimeMs} milliseconds, after which
  the cycle is repeated
\item
  {\bf DTC}     : a link to the corresponding DTC configuration(HB+EWM)'s
\item
  {\bf Enabled} and {\bf Status} : general fields of a configuration element
\end{itemize}

%%%%%%%%%%%%%%%%%%%%%%%%%%%%%%%%%%%%%%%%%%%%%%%%%%%%%%%%%%%%%%%%%%%%%%%%%%%%%%
\subsection {DTC configuration}

\begin{figure}[H]
{ \scriptsize
\begin{verbatim}
[local:test_025:S]mu2edaq09>ls -l DTC0
Key name                        Type    #Val  Size  Last Opn Mode Value
---------------------------------------------------------------------------
Link0                           DIR
Link1                           DIR
Link2                           DIR
Link3                           DIR
Link4                           DIR
Link5                           DIR
PcieAddress                     INT32   1     4     47h  0   RWD  0
EmulatesCFO                     INT32   1     4     2h   0   RWD  1
LinkMask                        UINT32  1     4     3h   0   RWD  1114385
JAMode                          INT32   1     4     2h   0   RWD  1
SampleEdgeMode                  INT32   1     4     2h   0   RWD  0
AutogenDRP                      INT32   1     4     47h  0   RWD  1
EnableClockMarkers              INT32   1     4     2h   0   RWD  0
EnableCFOLinkRX                 INT32   1     4     2h   0   RWD  1
DtcID                           INT32   1     4     47h  0   RWD  18
MacAddrByte                     INT32   1     4     47h  0   RWD  0
# comment_#_2 -------------     STRING  1     54    47h  0   RWD  ------------ tracker-specific 
Station                         INT32   1     4     47h  0   RWD  0
Plane                           INT32   1     4     47h  0   RWD  0
# comment_#_3 -------------     STRING  1     55    47h  0   RWD  --------- configurable element 
Enabled                         INT32   1     4     46h  0   RWD  1
Status                          INT32   1     4     47h  0   RWD  0
\end{verbatim}
  }
\caption{
  \label{figure:dtc_config}
  DTC configuration
}
\end{figure}


%%%%%%%%%%%%%%%%%%%%%%%%%%%%%%%%%%%%%%%%%%%%%%%%%%%%%%%%%%%%%%%%%%%%%%%%%%%%%%
\subsection {ROC configuration}

\begin{figure}[H]
{ \scriptsize
\begin{verbatim}
local:test_025:S]mu2edaq09>ls -l DTC0/Link0
Key name                        Type    #Val  Size  Last Opn Mode Value
---------------------------------------------------------------------------
DetectorElement -> /Mu2e/RunConfigurations/train_station/Tracker/Station_00/Plane_00/Panel_00
#--- configuration element      STRING  1     55    17h  0   RWD  ----------------------------
Enabled                         INT32   1     4     17h  0   RWD  1
Status                          INT32   1     4     17h  0   RWD  0
# -------- subsystem-specific   STRING  1     32    17h  0   RWD  --------- tracker ROC ------
RocDesignInfo                   STRING  1     67    46m  0   RWD  '000000000000000000000000000000000000000000000000000000434f52c7ef'
RocDeviceSerial                 STRING  1     35    46m  0   RWD  'dbf467da6e6387635683a91b34e60abf'
RocGitCommit                    STRING  1     14    46m  0   RWD  READ_DISABLED
\end{verbatim}
}
\caption{
  \label{figure:roc_config}
  ROC configuration
}
\end{figure}

\begin{itemize}
\item
  {\bf DetectorElement} : link to the detector element read out by this ROC.
  The detector element definition is subsystem-specific:
  \begin{itemize}
  \item
    tracker: a single panel
  \item
    calorimeter: 
  \item
    CRV: 
  \item
    STM:
  \item
    EXM: 
  \end{itemize}
\item
  {\bf Enabled, Status} : parameters of the configuration element
\item
  the rest parameters are subsystem-specific. For example, for the tracker they currently include ROC ID (device serial number),
  ROC design info - a label identifying the firmware build, and a git commit of the ROC firmware on github. 
\end{itemize}

%%%%%%%%%%%%%%%%%%%%%%%%%%%%%%%%%%%%%%%%%%%%%%%%%%%%%%%%%%%%%%%%%%%%%%%%%%%%%%
\subsection {Detector element configuration}

\begin{figure}[H]
{ \scriptsize
\begin{verbatim}
[local:test_025:S]Panel_00>ls -l
Key name                        Type    #Val  Size  Last Opn Mode Value
---------------------------------------------------------------------------
Name                            STRING  1     32    17h  0   RWD  MN016
# ------------- #1              STRING  1     55    47h  0   RWD  --------- configurable element ----
Enabled                         INT32   1     4     47h  0   RWD  1
Status                          INT32   1     4     47h  0   RWD  0
# ------------- #2              STRING  1     55    47h  0   RWD  --------- subsystem-specific ----
ID                              STRING  1     32    47h  0   RWD  1
\end{verbatim}
}
\caption{
  \label{figure:detector_element_config}
  Detector element configuration
}
\end{figure}

\begin{itemize}
\item
  {\bf Name}: name of the detector element, required
\item 
  {\bf Enabled, Status} : parameters of the configuration element
\item
  the rest parameters are subsystem-specific and will be added as needed, {\bf ID} is just an example
\end{itemize}

%%%%%%%%%%%%%%%%%%%%%%%%%%%%%%%%%%%%%%%%%%%%%%%%%%%%%%%%%%%%%%%%%%%%%%%%%%%%%%
\subsection{Global configuration frontend}
\label{sec:conf_frontends}

The configuration frontend, as it follows from its name, together with the
Sequencer, manages the hardware configuration at begin run.
It also registers the run transitions in the online run conditions database.


%%%%%%%%%%%%%%%%%%%%%%%%%%%%%%%%%%%%%%%%%%%%%%%%%%%%%%%%%%%%%%%%%%%%%%%%%%%%%% 
\subsection{Node frontends}
\label{sec:node_frontends}

\begin{itemize}
\item
  one monitoring/control frontend per DAQ server. Monitoring:
  \begin{itemize}
  \item
    2 DTC's with 6 ROCs per DTC
  \item
    ARTDAQ processes:
    \begin{itemize}
    \item
      2 boardreaders, N event builders, potentially a data logger, and a dispatcher
    \end{itemize}
  \item
    overall health: amount of free space available
  \end{itemize}
\item
  emulated CFO:
  \begin{itemize}
  \item
    currently : a separate frontend 
  \item 
    make the CFO frontend a separate thread of the node frontend
  \end{itemize}
\item
  external CFO frontend 
\item
  global control frontend:
\end{itemize}

%%%%%%%%%%%%%%%%%%%%%%%%%%%%%%%%%%%%%%%%%%%%%%%%%%%%%%%%%%%%%%%%%%%%%%%%%%%%%%
\subsection{ARTDAQ configuration}

ARTDAQ configuration is stored in ODB. It consists of two parts:
\begin{itemize}
\item
  Trigger Farm Manager (TFM) configuration
\item
  configuration of ARTDAQ subsystems
\item
  configuration of the artdaq processes on each farm node
\end{itemize}

A configuration of a single node is shown in Figure ~\ref{figure:artdaq_configuration}.

\begin{figure}[H]
  \begin{tikzpicture}
    \node[anchor=south west,inner sep=0] at (0,0.) {
      % \node[shift={(0 cm,0.cm)},inner sep=0,rotate={90}] at (0,0) {}
      \makebox[\textwidth][c] {
        \includegraphics[width=0.95\textwidth]{png/artdaq_configuration}
      }
    };
    % \node [text width=8cm, scale=1.0] at (14.5,0.5) {$\mu_B$, expected background mean};
    % \node [text width=8cm, scale=1.0, rotate={90}] at (1.5,7.5) { $S_{D}$, ``discovery'' signal strength  };
  \end{tikzpicture}
  \caption{
    \label{figure:artdaq_configuration}
    CFO configuration
  }
\end{figure}

It has
\begin{itemize}
\item
  parameters of the DTCs. There could be zero, one, or two DTCs on a farm node.
\item
  parameters of the artdaq processes running on that node.
  Configuration of the ARTDAQ boardreaders has links to the definitions
  of the DTCs they are reading. A boardreader need to know the PCIE address of the DTC
  it is interacting with. At start time, the boardreaders query that information
  from the ODB.
  
\item
  parameters of the node control frontend, including the configuration of the
  slow controls ("Frontend/Monitor")
\end{itemize}

%%%%%%%%%%%%%%%%%%%%%%%%%%%%%%%%%%%%%%%%%%%%%%%%%%%%%%%%%%%%%%%%%%%%%%%%%%%%%%
\subsection{DAQ-to-subdetectors interface}
The DAQ elements are "mapped" linked to the corresponding subdetector elements using ODB links,
as shown in Figure ~\ref{figure:daq_to_tracker_interface}.
\begin{figure}[H]
  \begin{tikzpicture}
    \node[anchor=south west,inner sep=0] at (0,0.) {
      % \node[shift={(0 cm,0.cm)},inner sep=0,rotate={90}] at (0,0) {}
      \makebox[\textwidth][c] {
        \includegraphics[width=0.95\textwidth]{png/daq_to_tracker_interface}
      }
    };
    % \node [text width=8cm, scale=1.0] at (14.5,0.5) {$\mu_B$, expected background mean};
    % \node [text width=8cm, scale=1.0, rotate={90}] at (1.5,7.5) { $S_{D}$, ``discovery'' signal strength  };
  \end{tikzpicture}
  \caption{
    \label{figure:daq_to_tracker_interface}
    CFO configuration
  }
\end{figure}

A DAQ node has links to the DTCs installed on that node, and the tracker DTC ROCs shown have
links to the definitions of the tracker panels they are reading.

%%% Local Variables:
%%% mode: latex
%%% TeX-master: t
%%% End:


%%%%%%%%%%%%%%%%%%%%%%%%%%%%%%%%%%%%%%%%%%%%%%%%%%%%%%%%%%%%%%%%%%%%%%%%%%%%%% 
\section{Detector configuration}

Configuration of the Mu2e detector is described by its subdetector configurations,
as shown in Figure~\ref{figure:configuration_top}.

For example, for the Mu2e tracker the levels of hierarchy look as follows:
tracker->stations->planes->panels.
Elements (substructures) are subdetector-specific, however element has
two mandatory fields: "enabled" and "status", used for configuration,
monitoring, and error reporting.

%%%%%%%%%%%%%%%%%%%%%%%%%%%%%%%%%%%%%%%%%%%%%%%%%%%%%%%%%%%%%%%%%%%%%%%%%%%%%% .

\subsubsection{Tracker} 


A top view of the tracker configuration
is shown in Figure~\ref{figure:tracker_config}.
\begin{itemize}
\item
  enabled = 0: the element (subtree) is considered present, but not used.
\item
  enabled = 1:
  \begin{itemize}
  \item
    status = 0 : the subsystem is OK
  \item
    status < 0 : the subsystem has a problem and an action is required
    The value of the status variable is the error code
  \item
    status > 0 : the subsystem has a warning-level problem, no immediate action
    is required
  \end{itemize}
\end{itemize}

\begin{figure}[H]
  \begin{tikzpicture}
    \node[anchor=south west,inner sep=0] at (0,0.) {
      % \node[shift={(0 cm,0.cm)},inner sep=0,rotate={90}] at (0,0) {}
      \makebox[\textwidth][c] {
        \includegraphics[width=0.95\textwidth]{png/tracker_config}
      }
    };
    % \node [text width=8cm, scale=1.0] at (14.5,0.5) {$\mu_B$, expected background mean};
    % \node [text width=8cm, scale=1.0, rotate={90}] at (1.5,7.5) { $S_{D}$, ``discovery'' signal strength  };
  \end{tikzpicture}
  \caption{
    \label{figure:tracker_config}
    Tracker: configuration levels from the station down to the panel, .
  }
\end{figure}


The next hierarchical levels of the tracker configuration, including the level of individual panels,
are shown in Figure~\ref{figure:station_config}.

\begin{figure}[H]
  \begin{tikzpicture}
    \node[anchor=south west,inner sep=0] at (0,0.) {
      % \node[shift={(0 cm,0.cm)},inner sep=0,rotate={90}] at (0,0) {}
      \makebox[\textwidth][c] {
        \includegraphics[width=0.95\textwidth]{png/station_config}
      }
    };
    % \node [text width=8cm, scale=1.0] at (14.5,0.5) {$\mu_B$, expected background mean};
    % \node [text width=8cm, scale=1.0, rotate={90}] at (1.5,7.5) { $S_{D}$, ``discovery'' signal strength  };
  \end{tikzpicture}
  \caption{
    \label{figure:station_config}
    Tracker configuration. The tracker, as well as each of the stations, has ``Enabled'' and
    ``Status'' fields.
  }
\end{figure}

%%%%%%%%%%%%%%%%%%%%%%%%%%%%%%%%%%%%%%%%%%%%%%%%%%%%%%%%%%%%%%%%%%%%%%%%%%%%%% 
\subsection{Calorimeter} 


%%%%%%%%%%%%%%%%%%%%%%%%%%%%%%%%%%%%%%%%%%%%%%%%%%%%%%%%%%%%%%%%%%%%%%%%%%%%%% 
\subsection{CRV} 


%%%%%%%%%%%%%%%%%%%%%%%%%%%%%%%%%%%%%%%%%%%%%%%%%%%%%%%%%%%%%%%%%%%%%%%%%%%%%% 
\subsection{STM} 




%%% Local Variables:
%%% mode: latex
%%% TeX-master: t
%%% End:

% \section{Interface to ARTDAQ}

The central part of the system is an online database (ODB)
implemented vis a shared memory segment

ODB stores all information about the running system -
the system configuration, status.

ODB has web-based interface, command line interface

Restricted functionality web server (MHTTPD) connects clients
to ODB. The web server doesn't execute applications, instead,
it simply updates the ODB.

Clients: frontends, communicating with the hardware and other
parts of the system.


%%% Local Variables:
%%% mode: latex
%%% TeX-master: t
%%% End:



%%%%%%%%%%%%%%%%%%%%%%%%%%%%%%%%%%%%%%%%%%%%%%%%%%%%%%%%%%%%%%%%%%%%%%%%%%%%%%
\section{Monitoring of the detector and its subsystems}

At run-time, the status of the detecotr is visualized by the system status tree.
The top page of the tree is shown in Figure~\ref{figure:mu2e_status_page}.

\begin{itemize}
\item
  one monitoring frontend per DAQ node, monitors the DTCs and the ARTDAQ processes.
  The frontend is responsible for setting status of the ROCs, boardreaders, and such.
  THe same frontend propagates the error status up the subdetector tree. 
  \item
  monitoring GUI - displays status of the detector. MIDAS-based javascript+HTML.
  Each element of the detector system, as described in ODB, in addition to other parameters
  has two mandatory ones: ``Enabled'' and ``Status''.
  \begin{itemize}
  \item 
    Setting Enabled=0 excludes the element from the configuration, in which case it will be
    shown in gray.
  \item
    Enabled=1 will result in the element shown in green (Status >= 0) or red (Status< 0) 
  \end{itemize}
\end{itemize}

\begin{figure}[H]
  \begin{tikzpicture}
    \node[anchor=south west,inner sep=0] at (0,0.) {
      % \node[shift={(0 cm,0.cm)},inner sep=0,rotate={90}] at (0,0) {}
      \makebox[\textwidth][c] {
        \includegraphics[width=0.95\textwidth]{png/mu2e_status_page}
      }
    };
    % \node [text width=8cm, scale=1.0] at (14.5,0.5) {$\mu_B$, expected background mean};
    % \node [text width=8cm, scale=1.0, rotate={90}] at (1.5,7.5) { $S_{D}$, ``discovery'' signal strength  };
  \end{tikzpicture}
  \caption{
    \label{figure:mu2e_status_page}
    Mu2e status page (prototype)
  }
\end{figure}

State of each enabled configurable element is monitored by a MIDAS frontend.
If a problem is detected, the status of the element is set to a negative
number, which value represents the status code of the problem.

A special configuration frontend propagates the status of the problem up the
configuration tree in ODB. For example, if a problem is detected with one of the
tracker panels, the status box corresponding to the tracker on the top monitoring
page will also become red.

Figure~\ref{figure:daq_status} shows the prototype of the DAQ monitoring page.
The screenshot has been takes after the end of one of the test runs.
One of the mu2edaq22 boardreaders is ``in the red'' because one the ROCs it
was reading timed out.

\begin{figure}[H]
  \begin{tikzpicture}
    \node[anchor=south west,inner sep=0] at (0,0.) {
      % \node[shift={(0 cm,0.cm)},inner sep=0,rotate={90}] at (0,0) {}
      \makebox[\textwidth][c] {
        \includegraphics[width=0.95\textwidth]{png/daq_status}
      }
    };
    % \node [text width=8cm, scale=1.0] at (14.5,0.5) {$\mu_B$, expected background mean};
    % \node [text width=8cm, scale=1.0, rotate={90}] at (1.5,7.5) { $S_{D}$, ``discovery'' signal strength  };
  \end{tikzpicture}
  \caption{
    \label{figure:daq_status}
    Prototype of the DAQ status page
  }
\end{figure}

After a problem with the ROC is cleared, and the readout resumes, the boardreader status
is set to zero by the monitoring frontend, and the color of the corresponding box changes to green.
The change is propagated up the configuration tree by the configuration frontend,
and the color of the DAQ monitoring box becomes green again.

%%%%%%%%%%%%%%%%%%%%%%%%%%%%%%%%%%%%%%%%%%%%%%%%%%%%%%%%%%%%%%%%%%%%%%%%%%%%%%
\section{Interaction with the hardware}

Interaction with the hardware is handled by the node frontends.

If manual interaction is required, a click on the DTC button on the DAQ monitoring
page shown in Figure~\ref{figure:daq_status} opens a DTC control page, prototype of which
is shown in  Figure~\ref{figure:dtc_control_page}. Parameters of the commands to be executed
are stored of ODB in the "/Mu2e/Commands" subtree and could be set interactively
via either a web-based interface or {\bf odbedit}. 

\begin{figure}[H]
  \begin{tikzpicture}
    \node[anchor=south west,inner sep=0] at (0,0.) {
      % \node[shift={(0 cm,0.cm)},inner sep=0,rotate={90}] at (0,0) {}
      \makebox[\textwidth][c] {
        \includegraphics[width=0.95\textwidth]{png/dtc_control_page}
      }
    };
    % \node [text width=8cm, scale=1.0] at (14.5,0.5) {$\mu_B$, expected background mean};
    % \node [text width=8cm, scale=1.0, rotate={90}] at (1.5,7.5) { $S_{D}$, ``discovery'' signal strength  };
  \end{tikzpicture}
  \caption{
    \label{figure:dtc_control_page}
    Prototype of the DTC control page
  }
\end{figure}
%%% Local Variables:
%%% mode: latex
%%% TeX-master: t
%%% End:

%%%%%%%%%%%%%%%%%%%%%%%%%%%%%%%%%%%%%%%%%%%%%%%%%%%%%%%%%%%%%%%%%%%%%%%%%%%%%%
\section{Operational procedures and details}

%%%%%%%%%%%%%%%%%%%%%%%%%%%%%%%%%%%%%%%%%%%%%%%%%%%%%%%%%%%%%%%%%%%%%%%%%%%%%%
\subsection{Execution of run transitions}

This section describes the sequence in which different run transitions
are executed by the system components.

\add{does OTSDAQ allow different execution sequences for
  different run transitions ? }

%%%%%%%%%%%%%%%%%%%%%%%%%%%%%%%%%%%%%%%%%%%%%%%%%%%%%%%%%%%%%%%%%%%%%%%%%%%%%%
\subsubsection{Begin Run}
\begin{itemize}
\item
  subdetector hardware configuration, uses the sequencer scripts
  When the system will become stable enough, the configuration functionality
  could be gradually moved to frontends.
\item
  request the next run number from the Mu2e run conditions database
  (MIDAS sequencer ==> config/scripts/get\_next\_run\_number.py) and stores
  it in ODB (rn-1)
  
\item
  start the run. At this point the detector is configured and ready to be read out.
  The TFM frontend starts the ARTDAQ processes , after which the CFO frontend 
  initiates new run plan.
  The sequence of the frontend execution is as follows:
  \begin{itemize}
  \item
    global config frontend - pre-begin run records the transition start
  \item
    DTC configuration frontends 
  \item
    TFM frontend - artdaq processes
  \item
    CFO frontend 
  \item
    global configuration frontend
  \end{itemize}
\end{itemize}

%%%%%%%%%%%%%%%%%%%%%%%%%%%%%%%%%%%%%%%%%%%%%%%%%%%%%%%%%%%%%%%%%%%%%%%%%%%%%%
\subsubsection{End Run}
\begin{itemize}
\item
  MIDAS end start
  \begin{itemize}
  \item
    global configuration frontend pre-end run callback: records
    the transition start
  \item
    CFO frontend stops executing the run plan
  \item
    TFM frontend - artdaq processes
  \item
    DTC frontends - artdaq processes
  \item
    global configuration frontend records the end of transition
  \end{itemize}
\end{itemize}


%%%%%%%%%%%%%%%%%%%%%%%%%%%%%%%%%%%%%%%%%%%%%%%%%%%%%%%%%%%%%%%%%%%%%%%%%%%%%%
\subsubsection{Pause and Resume Run}

Pause the run:

\begin{itemize}
\item
  configuration frontend: pre-transition callback: record the transition start
\item
  CFO frontend: disable output 
\item
  TFM frontend: pause the run
\item
  DTC frontends: do nothing 
\item
  configuration frontend: record the transition end
\end{itemize}

%%%%%%%%%%%%%%%%%%%%%%%%%%%%%%%%%%%%%%%%%%%%%%%%%%%%%%%%%%%%%%%%%%%%%%%%%%%%%%
\subsubsection{Resume Run}

Pause the run:

\begin{itemize}
\item
  configuration frontend: pre-transition callback: record the transition start
\item
  TFM frontend: pause the run
\item
  DTC frontends: do nothing 
\item
  CFO frontend: enable output 
\item
  configuration frontend: record the transition end
\end{itemize}

%%%%%%%%%%%%%%%%%%%%%%%%%%%%%%%%%%%%%%%%%%%%%%%%%%%%%%%%%%%%%%%%%%%%%%%%%%%%%%
\subsection{Host Names}

A DAQ host is typically simultaneously connected to several subnets and on
those subnets it may have different names.
For example, TCP traffic on a public network may be partially blocked,
in which case the DAQ communication has to be using a private network.

\begin{itemize}
\item
  "/Mu2e/ActiveRunConfiguration/DAQ/PublicSubnet"  :
  defines the host name used for labeling the host in ODB
\item
  "/Mu2e/ActiveRunConfiguration/DAQ/PrivateSubnet" :
  defines the host name used for defining the IPs
\end{itemize}

Usually the ``public'' names are shorter.


%%%%%%%%%%%%%%%%%%%%%%%%%%%%%%%%%%%%%%%%%%%%%%%%%%%%%%%%%%%%%%%%%%%%%%%%%%%%%%
\subsubsection{Naming conventions for ARTDAQ components}

It is assumed that :
\begin{itemize}
\item 
  artdaq boardreaders described in the configuration have names "br01", "br02", etc
\item 
  artdaq event builders have names "eb01", "eb02", etc
\item 
  artdaq data loggers have names "dl01", "dl02", etc
\item 
  artdaq dispatchers have names "ds01", "ds02", etc
\end{itemize}

This convention allows to use the ARTDAQ component names, as is, in the monitoring system.

%%%%%%%%%%%%%%%%%%%%%%%%%%%%%%%%%%%%%%%%%%%%%%%%%%%%%%%%%%%%%%%%%%%%%%%%%%%%%%
\subsection{Port assignment for ARTDAQ XMLRPC communication}

ARTDAQ convention: for a given partition, the first port number is base\_port=10000+1000*PARTITION
\begin{itemize}
\item
  boardreaders: base\_port+101 - base\_port+199\\
  expect the total number of boardreaders in the system to be < 100 \\
  The XMLRPC port of 'br01' is base\_port+101
\item
  filters (event builders) : base\_port+201 - base\_port+299
  The XMLRPC port of 'eb01' is base\_port+201
\item
  data loggers: base\_port+301 - base\_port+399
  The XMLRPC port of 'dl01' is base\_port+301
\item
  dispatchers: base\_port+401 - base\_port+499
  The XMLRPC port of 'ds01' is base\_port+301
\end{itemize}

%%% Local Variables:
%%% mode: latex
%%% TeX-master: t
%%% End:

%%%%%%%%%%%%%%%%%%%%%%%%%%%%%%%%%%%%%%%%%%%%%%%%%%%%%%%%%%%%%%%%%%%%%%%%%%%%%%
\section{Support considerations}

MIDAS has first been introduced about 30 years ago and has been used
by multiple experiments. It is used by the experiments at TRIUMF, PSI,
and CERN. Has been used by g-2. Currently is used by MEG, will be used by the Mu3e experiment which will have
a close to Mu2e structure of the DAQ.
It evolved over time, its architecture is advanced and uses modern software
technologies (HTML5, Javascript, Ajax).
It is supported on Windows, Macs and Linux.
Expect active support to continue over the lifetime of Mu2e.

\begin{itemize}
\item 
  MIDAS codebase is public and maintained at: https://bitbucket.org/tmidas/workspace/projects/PROJ
\item
  core developers : about 4-5 people
\item
  activity over the last 12 months: 100's of commits by the core developers/maintainers plus
  13 pull requests by 6 developers
\item 
  MIDAS forum - https://daq00.triumf.ca/elog-midas/Midas/
\end{itemize}




%%% Local Variables:
%%% mode: latex
%%% TeX-master: t
%%% End:


%%%%%%%%%%%%%%%%%%%%%%%%%%%%%%%%%%%%%%%%%%%%%%%%%%%%%%%%%%%%%%%%%%%%%%%%%%%%%% 
\section {Summary}

\begin{itemize}
\item 
  The MIDAS+ARTDAQ solution for the Mu2e DAQ has been extensively prototyped.
\item
  the solution is significantly simpler for both the users and the developers
\item
  its current functionality, in many respects, already exceeds that
  of the current OTSDAQ-based system  
\item
  collaboration members, including students, are contributing
  to the development process. 
\item
  new collaboration members joining Mu2e from other muon experiment will
  likely have an experience of operating a MIDAS-based DAQ
\end{itemize}

Based on our assessment, a combination of MIDAQ and  ARTDAQ
provides the best solution for the Mu2e DAQ.

We therefore propose to exercise this solution and demonstrate
its functionality in the upcoming GR4.

%%%%%%%%%%%%%%%%%%%%%%%%%%%%%%%%%%%%%%%%%%%%%%%%%%%%%%%%%%%%%%%%%%%%%%%%%%%%%% 
%
%%%%%%%%%%%%%%%%%%%%%%%%%%%%%%%%%%%%%%%%%%%%%%%%%%%%%%%%%%%%%%%%%%%%%%%%%%%%%%
\newpage
\bibliographystyle{unsrtnat}
\bibliography{clfv,mu2e_internal_notes,daq}

% \include{appendix_a}
\include{appendix_b}

\end{document}
