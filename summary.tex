%%%%%%%%%%%%%%%%%%%%%%%%%%%%%%%%%%%%%%%%%%%%%%%%%%%%%%%%%%%%%%%%%%%%%%%%%%%%%% 
\section {Summary}

\begin{itemize}
\item 
  The MIDAS+ARTDAQ solution for the Mu2e DAQ has been extensively prototyped
  over the last several months.
\item
  the functionality of the described system, in many aspects,
  already exceeds that of the OTSDAQ-based system  
\item
  the system is significantly simpler than the OTSDAQ-based solution,
  for both users and developers
\item
  collaboration members, including students, are contributing
  to the development process. 
\item
  new collaboration members joining Mu2e from other muon experiments
  will more likely be familiar with a MIDAS-based DAQ than with
  any other DAQ system
\end{itemize}

Based on our assessment, a combination of MIDAS and ARTDAQ provides
the solution for the Mu2e DAQ which the collaboration will be able
to maintain and operate over the course of the Mu2e data taking.

\vspace{0.2in}
We therefore propose to exercise the MIDAS+ARTDAQ based DAQ system
in the upcoming GR4 to further test its functionality and determine
next steps. 

%%% Local Variables:
%%% mode: latex
%%% TeX-master: t
%%% End:
