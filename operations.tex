%%%%%%%%%%%%%%%%%%%%%%%%%%%%%%%%%%%%%%%%%%%%%%%%%%%%%%%%%%%%%%%%%%%%%%%%%%%%%%
\section{Operational procedures and details}

%%%%%%%%%%%%%%%%%%%%%%%%%%%%%%%%%%%%%%%%%%%%%%%%%%%%%%%%%%%%%%%%%%%%%%%%%%%%%%
\subsection{Execution of run transitions}

This section describes the sequence in which different run transitions
are executed by the system components.

\add{does OTSDAQ allow different execution sequences for
  different run transitions ? }

%%%%%%%%%%%%%%%%%%%%%%%%%%%%%%%%%%%%%%%%%%%%%%%%%%%%%%%%%%%%%%%%%%%%%%%%%%%%%%
\subsubsection{Begin Run}
\begin{itemize}
\item
  subdetector hardware configuration, uses the sequencer scripts
  When the system will become stable enough, the configuration functionality
  could be gradually moved to frontends.
\item
  request the next run number from the Mu2e run conditions database
  (MIDAS sequencer ==> config/scripts/get\_next\_run\_number.py) and stores
  it in ODB (rn-1)
  
\item
  start the run. At this point the detector is configured and ready to be read out.
  The TFM frontend starts the ARTDAQ processes , after which the CFO frontend 
  initiates new run plan.
  The sequence of the frontend execution is as follows:
  \begin{itemize}
  \item
    global config frontend - pre-begin run records the transition start
  \item
    DTC configuration frontends 
  \item
    TFM frontend - artdaq processes
  \item
    CFO frontend 
  \item
    global configuration frontend
  \end{itemize}
\end{itemize}

%%%%%%%%%%%%%%%%%%%%%%%%%%%%%%%%%%%%%%%%%%%%%%%%%%%%%%%%%%%%%%%%%%%%%%%%%%%%%%
\subsubsection{End Run}
\begin{itemize}
\item
  MIDAS end start
  \begin{itemize}
  \item
    global configuration frontend pre-end run callback: records
    the transition start
  \item
    CFO frontend stops executing the run plan
  \item
    TFM frontend - artdaq processes
  \item
    DTC frontends - artdaq processes
  \item
    global configuration frontend records the end of transition
  \end{itemize}
\end{itemize}


%%%%%%%%%%%%%%%%%%%%%%%%%%%%%%%%%%%%%%%%%%%%%%%%%%%%%%%%%%%%%%%%%%%%%%%%%%%%%%
\subsubsection{Pause and Resume Run}

Pause the run:

\begin{itemize}
\item
  configuration frontend: pre-transition callback: record the transition start
\item
  CFO frontend: disable output 
\item
  TFM frontend: pause the run
\item
  DTC frontends: do nothing 
\item
  configuration frontend: record the transition end
\end{itemize}

%%%%%%%%%%%%%%%%%%%%%%%%%%%%%%%%%%%%%%%%%%%%%%%%%%%%%%%%%%%%%%%%%%%%%%%%%%%%%%
\subsubsection{Resume Run}

Pause the run:

\begin{itemize}
\item
  configuration frontend: pre-transition callback: record the transition start
\item
  TFM frontend: pause the run
\item
  DTC frontends: do nothing 
\item
  CFO frontend: enable output 
\item
  configuration frontend: record the transition end
\end{itemize}

%%%%%%%%%%%%%%%%%%%%%%%%%%%%%%%%%%%%%%%%%%%%%%%%%%%%%%%%%%%%%%%%%%%%%%%%%%%%%%
\subsection{Host Names}

A DAQ host is typically simultaneously connected to several subnets and on
those subnets it may have different names.
For example, TCP traffic on a public network may be partially blocked,
in which case the DAQ communication has to be using a private network.

\begin{itemize}
\item
  "/Mu2e/ActiveRunConfiguration/DAQ/PublicSubnet"  :
  defines the host name used for labeling the host in ODB
\item
  "/Mu2e/ActiveRunConfiguration/DAQ/PrivateSubnet" :
  defines the host name used for defining the IPs
\end{itemize}

Usually the ``public'' names are shorter.


%%%%%%%%%%%%%%%%%%%%%%%%%%%%%%%%%%%%%%%%%%%%%%%%%%%%%%%%%%%%%%%%%%%%%%%%%%%%%%
\subsubsection{Naming conventions for ARTDAQ components}

It is assumed that :
\begin{itemize}
\item 
  artdaq boardreaders described in the configuration have names "br01", "br02", etc
\item 
  artdaq event builders have names "eb01", "eb02", etc
\item 
  artdaq data loggers have names "dl01", "dl02", etc
\item 
  artdaq dispatchers have names "ds01", "ds02", etc
\end{itemize}

This convention allows to use the ARTDAQ component names, as is, in the monitoring system.

%%%%%%%%%%%%%%%%%%%%%%%%%%%%%%%%%%%%%%%%%%%%%%%%%%%%%%%%%%%%%%%%%%%%%%%%%%%%%%
\subsection{Port assignment for ARTDAQ XMLRPC communication}

ARTDAQ convention: for a given partition, the first port number is base\_port=10000+1000*PARTITION
\begin{itemize}
\item
  boardreaders: base\_port+101 - base\_port+199\\
  expect the total number of boardreaders in the system to be < 100 \\
  The XMLRPC port of 'br01' is base\_port+101
\item
  filters (event builders) : base\_port+201 - base\_port+299
  The XMLRPC port of 'eb01' is base\_port+201
\item
  data loggers: base\_port+301 - base\_port+399
  The XMLRPC port of 'dl01' is base\_port+301
\item
  dispatchers: base\_port+401 - base\_port+499
  The XMLRPC port of 'ds01' is base\_port+301
\end{itemize}

%%% Local Variables:
%%% mode: latex
%%% TeX-master: t
%%% End:
