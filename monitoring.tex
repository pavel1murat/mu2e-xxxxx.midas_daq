%%%%%%%%%%%%%%%%%%%%%%%%%%%%%%%%%%%%%%%%%%%%%%%%%%%%%%%%%%%%%%%%%%%%%%%%%%%%%%
\section{Monitoring}

Components of the monitoring system:

\begin{itemize}
\item
  status of the running system is represented by the status tree. The top page
  of the tree is shown in Figure~\ref{figure:mu2e_status_page}.
\item
  one monitoring frontend per DAQ node, monitors the DTCs and the ARTDAQ processes.
  The frontend is responsible for setting status of the ROCs, boardreaders, and such.
  THe same frontend propagates the error status up the subdetector tree. 
  \item
  monitoring GUI - displays status of the detector. MIDAS-based javascript+HTML.
  Each element of the detector system, as described in ODB, in addition to other parameters
  has two mandatory ones: ``Enabled'' and ``Status''.
  \begin{itemize}
  \item 
    Setting Enabled=0 excludes the element from the configuration, in which case it will be
    shown in gray.
  \item
    Enabled=1 will result in the element shown in green (Status >= 0) or red (Status< 0) 
  \end{itemize}
\end{itemize}

\begin{figure}[H]
  \begin{tikzpicture}
    \node[anchor=south west,inner sep=0] at (0,0.) {
      % \node[shift={(0 cm,0.cm)},inner sep=0,rotate={90}] at (0,0) {}
      \makebox[\textwidth][c] {
        \includegraphics[width=0.95\textwidth]{png/mu2e_status_page}
      }
    };
    % \node [text width=8cm, scale=1.0] at (14.5,0.5) {$\mu_B$, expected background mean};
    % \node [text width=8cm, scale=1.0, rotate={90}] at (1.5,7.5) { $S_{D}$, ``discovery'' signal strength  };
  \end{tikzpicture}
  \caption{
    \label{figure:mu2e_status_page}
    Mu2e status page (prototype)
  }
\end{figure}


%%% Local Variables:
%%% mode: latex
%%% TeX-master: t
%%% End:
