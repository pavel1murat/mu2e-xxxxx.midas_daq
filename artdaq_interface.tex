\section{Interface to ARTDAQ}

ARTDAQ processes are controlled by the Trigger Farm Manager (TFM) frontend,
tfm\_launch\_fe.py.

\begin{itemize}
\item 
  TFM reads its initial settings from the "/Mu2e/ActiveRunConfiguration/DAQ/Tfm"
  ODB configuration path.
\item
  FCL files of ARTDAQ processes are stored in \$MU2E\_DAQ\_DIR/config/\$run\_configuration
  subdirectory assumed to be the same (network mounted) on all DAQ nodes.
  The FCL files could be regenerated manually based on the run configuration and the trigger table.
\item
  Archival mechanism: the FCL files used for a given run are stored
  in \$DAQ\_OUTPUT\_TOP/run\_records/\$run\_number area of the node where the TFM frontend is running.
\item
  the log files of ARTDAQ processes running on a given node are stored in \$DAQ\_OUTPUT\_TOP/logs
  directory of that node
\end{itemize}

%%%%%%%%%%%%%%%%%%%%%%%%%%%%%%%%%%%%%%%%%%%%%%%%%%%%%%%%%%%%%%%%%%%%%%%%%%%%%%
\subsection{Naming conventions for ARTDAQ components}

It is assumed that :
\begin{itemize}
\item 
  artdaq boardreaders described in the configuration have names "br01", "br02", etc
\item 
  artdaq event builders have names "eb01", "eb02", etc
\item 
  artdaq data loggers have names "dl01", "dl02", etc
\item 
  artdaq dispatchers have names "ds01", "ds02", etc
\end{itemize}

This convention allows to use component names, as they are, in the monitoring system.

%%%%%%%%%%%%%%%%%%%%%%%%%%%%%%%%%%%%%%%%%%%%%%%%%%%%%%%%%%%%%%%%%%%%%%%%%%%%%%
\subsection{Port assignment for ARTDAQ XML-RPC communication}
\begin{itemize}
\item
  base\_port: 10000+1000*partition\_number
\item 
  TFM : base\_port
\item 
  MIDAS node monitoring frontend: base\_port+11;
\item 
  artdaq boardreaders: base\_port+101 and base\_port+102
\item 
  artdaq event builders : base\_port + 201 - base\_port+300;
\item 
  artdaq data loggers  : base\_port + 301 - base\_port+400;
\item 
  artdaq dispatchers : base\_port + 401 - base\_port+500
\end{itemize}


%%% Local Variables:
%%% mode: latex
%%% TeX-master: t
%%% End:


